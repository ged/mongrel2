\chapter{Deploying}

Mongrel2 is designed to be easy to deploy and \emph{automate} the deployment.
This is why it uses \href{http://www.sqlite.org/}{SQLite} to store the configuration,
but \shell{m2sh} as an interface to creating the configuration.  Doing this lets
you access the configuration using any language that works for you, augment it,
alter it, migrate it, and automate it.

In this chapter I'm going to show you how to make a basic configuration using
m2sh and all the commands that are available.  You'll learn how the configuration
system is structured so that you know what goes where, but in the end it's just
a simple storage mechanism.

\begin{aside}{Apparently SQL Inspires FUD}
When I first started talking about Mongrel2 I said I'd store the configuration
in SQLite and do a Model-View-Controller kind of design.  Immediately people who
can't read flipped out and though this meant they'd be back in "windows registry hell"
but with SQL as their only way to access it.  They thought that they'd be stuck writing
configurations with SQL.  That SQL couldn't possibly configure a web server.

They were wrong on many levels.  Nobody was \emph{ever} going to make \emph{anyone} use
SQL.  That was repeated over and over, but again, people don't read and love spreading
FUD.  The SQLite config database is nothing like the Windows Registry.  No other web
server really uses a true hierarchy, they just cram a relational model into a weirdo
configuration format.  The real goal was to make a web server that was easy to manage from
\emph{any} language, and then give people a nice tool to get their job done without
having to ever touch SQL.  \emph{EVER!}

In the end, what we got despite all this fear mongering is a bad ass configuration
tool and a design that is simple, elegant, and works fantastic.  If you read that
Mongrel2 uses SQLite and though this was weird, well welcome to the future.  Sometimes
it's weird out here (even though, Postfix has been doing this for a decade or more).
\end{aside}


\section{Model-View-Controller}


\section{Trying m2sh}


\section{A Simple Configuration File}


\section{A More Complex Example}


\section{How It's Structured}


\section{More Advanced m2sh}



\section{Writing Your Own m2sh}


