\chapter{Управление}

Mongrel2 создан так, чтобы его можно было легко развёртывать, более
того --- \emph{автоматизировать} развёртывание. Именно поэтому он
использует \href{http://www.sqlite.org/}{SQLite}, чтобы хранить
настройки, и утилиту m2sh в качестве интерфейса к этой базе настроек.
По сути, вы можете управлять этой базой из любого языка.

В этой главе я продемонстрирую создание базовой конфигурации в m2sh, а
также другие доступные команды. Вы узнаете внутренню структуру
конфигурационной системы, которая в общем является простым и удобным
механизмом хранения данных.

\begin{aside}{SQL вводит в заблуждение}

Когда я впервые упомянул Mongrel2, я сказал, что буду хранить
настройки с помощью SQLite и создам MVC архитектуру.  Те, кто не умеют
читать, начали биться в истерике и кричать, что это равносильно
``реестру Windows'', только для доступа к данным используется язык
запросов.  Они думали, что придётся конфигурировать сервер с помощью
SQL. Естественно, звучит, мягко говоря, непривычно.

Всё это --- большое заблуждение. \emph{Никто никого} не заставляет
использовать SQL.  Я повторял это снова и снова, но многие не слышали
этого.  SQLite не имеет ничего общего с реестром Windows. Моя
изначальная цель --- создать сервер, который можно настроить из
\emph{любого языка программирования}, а затем предоставить удобный
инструмент, чтобы люди вообще \emph{навсегда} забыли про SQL.

В итоге, что мы имеем, несмотря на всю суету? Офигительный инструмент
для конфигурирования сервера с простым и элегантным дизайном, который,
вдобавок к этому, отлично работает. Если союз Mongrel2 и SQLite
кажется вам необычным, то добро пожаловать в будущее --- иногда оно
очень и очень странное (хотя Postfix делает то же самое на протяжении
десяти лет).

\end{aside}


\section{MVC}

Когда вы слышите MVC\footnote{Model-View-Controller}, вы думаете о
веб-приложениях, где этот паттерн проектирования стал стандартом. Суть его в
разделении системы на уровни по функциональному признаку. В такой системе
\emph{модель} хранит данные, \emph{контроллер} реализует бизнес-логику, а
взаимодействие с пользователем происходит посредством \emph{вида}.

Сила такой архитектуры в том, что все компоненты достаточно независимы
и автономны и их легко адаптировать к новым требованиям. Однако они
перестают быть таковыми, как только вы начинаете их смешивать.  Нет
какой-либо строгой математически обоснованной теории, подтверждающей
это. Но есть многолетний опыт сообщества, который подсказывает:
смешивать уровни --- плохая идея, иначе возникнут проблемы. Например,
вы раскидали бизнес-логику (контроллер) по веб-страницам (вид) и позже
обнаруживаете, что её достаточно сложно изменять.  Или не можете
изменить схему базы (модель), потому что это противоречит хранимым
процедурам (контроллер).

Mongrel2 нуждался в механизме автоматизации: ведь это одна из ключевых
функций сервера, если не самая ключевая. Сервер является контроллером,
т.е.  он реализует бизнес-логику и придаёт смысл модели. В качестве
модели был выбран SQLite. Он идеально подходит: внедряемый движок,
легко доступный из любого языка программирования, портируемый,
маленький, быстрый и т.п. Дело оставалось за видом --- так и появилась
утилита m2sh.

m2sh (доступная из \file{tools/m2sh}) --- это \emph{вид}, интерфейс к модели,
которая хранит конфигурацию сервера. Эта утилита только пример; вы можете
создать свой вид на любимом языке программирования. Это может быть всё что
угодно --- скрипт для автоматического развёртывания сервера, веб-интерфейс,
скрипт мониторинга, и т.п.

В общем и целом всё просто: если вы просто хотите выполнять базовые
операции с сервером --- запускать, останавливать, создавать и
загружать конфигурации --- используйте m2sh. Если же вам нужно что-то
более продвинутое, то изучите схему базы и пробуйте. Структура базы
достаточно хорошо отражает внутренню структуру сервера, так что если
вы поймёте её, то вы в принципе поймёте как работает Mongrel2.

\section{Пробуем m2sh}

Пробовать будем на конфигурационном файле, который используется в
юнит-тестах. Для разогрева выполним самые базовые команды:

\begin{code}{m2sh в действии}
\begin{lstlisting}
> m2sh servers -db tests/config.sqlite
> m2sh hosts -db tests/config.sqlite -server test
> m2sh running -db tests/config.sqlite -name test
> m2sh start -db tests/config.sqlite -host localhost
\end{lstlisting}
\end{code}

Вы только что сделали следующее: вывели настройки на экран; вывели
список серверов, затем список хостов; проверили, запущен ли сервер
(он не запущен); и запустили сервер.

Чтобы узнать больше об этой команде, вызовите справку --- \shell{m2sh
help} или \shell{m2sh help command}.

Теперь попробуем запустить-остановить сервер в sudo-режиме и
перезагрузить настройки (нажмите CTRL+c, чтобы прервать предыдущую
команду):

\begin{code}{Запускаем, останавливаем, перезагружаем}
\begin{lstlisting}
> m2sh start -db tests/config.sqlite -host localhost -sudo
> tail logs/error.log
> m2sh reload -db tests/config.sqlite -host localhost
> tail logs/error.log
> curl http://localhost:6767/
> tail logs/error.log
> m2sh running -db tests/config.sqlite -host localhost
> m2sh stop -db tests/config.sqlite -host localhost
\end{lstlisting}
\end{code}

Круто, да? Используя этот маленький инструмент можно полностью
контролировать сервер. И не нужно ковыряться в конфигурационных
файлах. Однако неплохо бы знать как всё это работает.

\subsection{А что произошло?}

А произошло следующее --- вы выполнили большинство из возможных
операций над базой настроек. Ниже приведено детальное
объяснение всех запущенных команд:

\begin{enumerate}
\item Вызов \shell{m2sh start} с опцией \shell{-sudo} транслируется в
    \shell{sudo mongrel2 tests/config.sqlite localhost} и запускает
    сервер.
\item Mongrel2 запущен и работает в фоновом процессе, как и подобает
    серверу. Но предварительно текущую директорию он сделал рутовой
    \footnote{С помощью команды chroot}, а текущего пользователя ---
    владельцем процесса. Вызовите \shell{ps aux}, чтобы убедиться, что
    сервер действительно запущен.
\item Смотрим логи в logs/error.log: там куча отладочной информации.
    Но обратите внимание на сообщения --- детальные и адекватные.
\item Горячая перезагрузка: процесс mongrel2 заново считывает и загружает
    настройки.
\item Однако из-за небольшого бага перезагрузка не осуществляется
    немедленно --- надо подождать до следующего запроса, который
    эмулируется с помощью \shell{curl http://localhost:6767/}.
\item Чтобы убедиться, что перезагрузка прошла успешно, смотрим логи и
    выполняем \shell{m2sh running}. Эта команда считывает с базы настроек
    расположение PID-файла (\file{run/mongrel2.pid}) и проверяет, запущен
    ли этот процесс.
\item Наконец, останавливаем сервер. Замечательная вещь в том, что,
    поскольку в момент запуска текущий пользователь стал владельцем
    процесса, вы можете остановить этот процесс не прибегая sudo.
\end{enumerate}

\section{Базовый конфигурационный файл}

Для создания новой базы настроек вы напишите конфигурационный файл.
Файл похож на скрипт на Питоне; изначально он и был таким скриптом,
поскольку первая версия утилиты m2sh была на написана на Питоне (и
сейчас находится в \file{examples/python}). Чуть позже она была
переписана на Си.  Несмотря на то, что она была полностью переделана,
мы сохранили первоначальный формат. И даже немного его упростили,
сделав запятые необязательными во многих местах.

Сначала надо загрузить этот файл в чистую базу с помощью команды
\shell{m2sh load}. Для нашего примера воспользуемся файлом
\file{examples/configs/sample.conf}:

\begin{code}{Простой маленький скрипт настроек}
  \lstinputlisting[language=Python]{../../../examples/configs/sample.conf}
\end{code}

Если Python вам не знаком, то код может показаться необычным, но на
самом деле он довольно прост. Мы вскоре вернёмся к нему, а пока
выполните следующее:

\begin{code}{Загружаем скрипт}
\begin{lstlisting}
m2sh load -config examples/configs/sample.conf
ls -l config.sqlite
m2sh servers
m2sh hosts -server test
m2sh start -server test
\end{lstlisting}
\end{code}

Обратите внимание, что по умолчанию база записывается в файл
\file{config.sqlite}, поэтому можно её явно не указывать в команде.
Конфигурационным файлом по умолчанию является \file{mongrel2.conf} (но
поскольку в нашем примере используется другой файл, мы его явно
передаём).

Последовательность этих команд выполняет следующее:

\begin{enumerate}
\item Создаёт чистую базу настроек \file{simple.sqlite} и загружает
в неё скрипт \file{sample.conf}.
\item Выводит список серверов из базы.
\item Выводит список хостов со всеми маршрутами.
\item Запускает сервер.
\end{enumerate}

Полагаю, у вас начала складываться картина того, как создать новую
конфигурацию --- использовать m2sh, чтобы загрузить скрипт и
сгенерировать .sqlite, с которым непосредственно работает Mongrel2.

\section{Структура базы настроек}

База настроек структурирована следующим образом:

\begin{description}
  \item[Server] Является корнем в иерархии; их может быть несколько, несмотря на то, что 
    Mongrel2 одновременно может запустить один их них.
  \begin{description}
    \item[Host] Внутри сервера --- хосты, т.е. доменные имена, на
      которые должен отвечать веб-сервер. Их тоже может быть несколько.
    \begin{description}
      \item[Route] Внутри хостов находятся маршруты, которые
        разруливают приходящие запросы в зависимости от URL. Внутри
        маршрута может быть \ident{Dir}, \ident{Handler} или \ident{Proxy}.
      \begin{description}
        \item[Dir] Самый простой маршрут --- говорит серверу, что надо
          выдавать файлы из директории. Mongrel2 поддерживает 304-й статус,
          ETag, типы данных по умолчанию, и вообще всё необходимое для
          обслуживания статических файлов.
        \item[Proxy] Прокси берёт запрос и перенаправляет его на
          другой http-сервер. Mongrel2 держит соединения открытыми в keep-alive
          режиме, если браузер поддерживает их.
        \item[Handler] Это самая лучшая часть Mongrel2. Обработчик
          получает http-запрос, заворачивает его в ZeroMQ сообщение и передаёт
          его в бэкенд, но делает это асинхронно.
      \end{description}
    \end{description}
  \end{description}
\end{description}

Каждый из этих ``объектов'' имеет свой набор атрибутов для более
тонкой настройки и большинству из этих атрибутов присваиваются
достаточно адекватные значения по умолчанию.

\subsection{Server}

Этот раздел хранит настройки, характерные для любого сервера: какой
порт слушать, куда писать логи и т.п.

\begin{description}
\item[uuid] Уникальный идентификатор сервера внутри вашей
    инфраструктуры. Это может быть любая строка, но традиционно
    используется UUID --- универсальный уникальный идентификатор, который
    можно сгенерировать с помощью \shell{m2sh uuid}.
\item[chroot] Директория, которую Mongrel2 делает рутовой.
\item[access\_log] Путь к лог-файлу запросов.
    Путь задаётся относительно \emph{рутовой директории}. Обычно
    начинается с `/'. Убедитесь, что этот и другие подобные файлы
    недоступны из браузера.
\item[error\_log] Путь к лог-файлу ошибок.
\item[pid\_file] Путь к pid-файлу.
\item[default\_host] Сервер может иметь произвольное количество
    хостов, но среди них должен быть один по умолчанию. С помощью этого
    хоста удобно ссылаться на весь сервер.
\item[port] Порт, который должен слушать сервер.
\end{description}


\subsection{Host}

Раздел Host хранит информацию о каждом хосте, доступном из сервера.

\begin{description}
\item[name] Имя хоста.
\item[matching] Паттерн, который используется, чтобы сопоставить с
    хостом в заголовке для принятия решения о дальнейшей маршрутизации запроса.
\item[server] Используйте этот атрибут, если хотите назначить сервер отдельно.
\item[maintenance] Настройка для будущего использования. Если сервер
    некоторое время недоступен из-за проведения профилактических работ,
    этот атрибут будет указывать на временную страничку, которая будет
    уведомлять об этом посетителей.
\item[routes] Хеш-таблица, которая хранит данные по маршрутизации запросов.
\end{description}


\subsection{Route}

Route --- или маршрут --- рабочая лошадка всей системы. Маршруты
используют довольно интересный, но простой код, чтобы транслировать
путь в http-заголовке в один из пунктов назначения на сервере:
\ident{Handler}, \ident{Dir}, \ident{Proxy}.

\begin{description}
\item[path] \emph{Паттерн} соответствия маршруту. Mongrel2 использует
    упрощённую версию системы сопоставления паттернов из Lua.
\item[reversed] Определяет, является ли паттерн реверсивным. Этот
    атрибут полезен для сравнения расширений файлов, хостов и пр., где
    окончание фактически является префиксом. Обычно этот атрибут игнорируется.
\item[host] Используется, чтобы задать хост вручную.
\item[target] Пункт назначения запроса, т.е. что-либо из списка:
\ident{Dir}, \ident{Handler}, \ident{Proxy}.
\end{description}

Позже вы узнаете о системе сравнения паттернов, используемой в
сервере, но по сути это упрощённая версия регулярных выражений c
некоторыми добавлениями. При конфигурировании маршрута пишется
что-то вроде \file{/images/(.*.jpg)}. Всё, что идёт до первой скобки,
используется как префикс (который легко сравнить); в то время как всё
после скобки --- собственно паттерн. Когда приходит запрос, Mongrel2
ищет самый длинный подходящий префикс и проверяет паттерн. Если
паттерн подходит, то запрос идёт дальше. В противно случае --- 404.

\subsection{Dir}

\ident{Dir} --- инструктирует веб-сервер обслуживать файлы из
директории. Поддерживает If-Modified-Since, ETag и прочие странные
механизмы кэширования согласно протокола HTTP. Также поддерживает
индексные файлы и типы данных по умолчанию.

\begin{description}
\item[base] Базовая директория \emph{внутри рутовой}, из которой будут
    выдаваться файлы.
\item[index\_file] Файл по умолчанию; он выдаётся, если в запросе файл явно
    не указан. \ident{Dir} также делает редирект, если запрашивается
    директория, а на конце нет слэша.
\item[default\_ctype] Content-Type по умолчанию; используется, если в
    таблице типов нет подходящего MIMEType.
\end{description}

На данный момент это все доступные параметры, но со временем мы будем
добавлять больше настроек, чтобы управлять работой
\ident{Dir}-маршрутов.

\subsection{Proxy}

Прокси добавлены в Mongrel2 для того, чтобы вы могли использовать
существующую инфраструктуру. Мы приложили все усилия, чтобы сделать
систему проксирования быстрой и надёжной внутри Mongrel2, тем не
менее, она не конкурент обработчикам ZeroMQ. Я вижу один сценарий
развития --- вы добавляете прокси, чтобы Mongrel2 перенаправлял
запросы на ваши текущие серверы; но постепенно вы переводите всё на
обработчики.

\begin{description}
\item[addr] DNS-адрес сервера.
\item[port] Порт подключения.
\end{description}

Mongrel2 распарсивает запросы, которые попадают под проксирование,
поэтому ваши серверы должны получать эти запросы в нормальном виде.
Ответы же от прокси передаются как есть в браузер.

\subsection{Handler}

Ну вот, мы подошли к самой интересной части:
\ident{ZeroMQ-обработчикам}, которые получают асинхронные запросы от
Mongrel2. Для их конфигурирования нужно использовать ZeroMQ-синтаксис.
Плюс его в том, что разные транспортные протоколы --- UDP, TCP, Unix,
PGM --- одинаково настраиваются. Мы в основном тестировали TCP.

\begin{description}
\item[send\_spec] Спецификация отправителя в формате 0MQ. Например,
    строка ``tcp://127.0.0.1:9999'' говорит, что надо соединяться с
    сервером по адресу 127.0.0.1 на порт 9999, используя протокол TCP.
    Отправители используют PUSH-сокеты, чтобы обработчики получали запросы
    в стиле round-robin.
\item[send\_ident] Этот идентификатор (обычно UUID) используется для
    регистрации отправителя. Если ваш сервер рухнет по каким-то причинам,
    этот идентификатор используется, чтобы восстановить необработанные
    сообщения.
\item[recv\_spec] Подобно спецификации отправителя, только для
    получения ответов от обработчиков. Получатели используют SUB-сокеты,
    т.е. подписчики. Это позволяет кластеру из серверов Mongrel2 слушать
    ответы, но только одному из них с подходящим recv\_ident --- реагировать.
\item[recv\_ident] Ещё один UUID. Полезен, если необходимо подписаться
    и слушать ответы. В ответах обработчики передают send\_ident ---
    идентификатор отправителя. Так что у вас два варианта: присвоить
    recv\_ident то же значение, что и send\_ident и таким образом
    подписаться на ответы; либо ничего не присваивать и не подписываться.
\end{description}

Примечательный факт --- веб-серверу абсолютно \emph{безралично} где
расположены бэкенды. Вы обратили внимание на то, что не нужно
прописывать кластеры из прокси, или методы выбора прокси-сервера, или
ещё что-либо в этом духе? Всё что нужно определить --- две конечные
точки и их идентификаторы. Это потому что Mongrel2 \emph{связывается}
с сокетами и слушает. Mongrel2 не подключается к бэкэндам; напротив
--- бэкенды подключаются к нему. Это значит, что если вам нужно ещё 10
обработчиков, чтобы справиться с нагрузкой, вы просто их запускаете и
всё. Не надо перегружать или заново конфигурировать веб-сервер.

\subsection{Другое}

Есть также объекты/таблицы \ident{MIMEType}, \ident{Log}, и
\ident{Setting}. Но они не нужны, чтобы понять структуру сервера,
поэтому мы вернёмся к ним позже.

\section{Более сложный пример}

Теперь, вооружившись всеми необходимыми знаниями о структуре Mongrel2,
можно взглянуть на более сложный пример. Я буду комментировать, вы ---
сопоставлять мои комментарии с кодом.  Файл с примером находится
здесь: \file{examples/configs/mongrel2.conf}:

\begin{code}{Скрипт настроек mongrel2.org}
  \lstinputlisting[language=Python]{../../../examples/configs/mongrel2.conf}
\end{code}

Если вы ещё не догадались, этот скрипт используется для сайта
\url{http://mongrel2.org}. И вот что в нём есть:

\begin{enumerate}
\item Простой сервер и в нём mongrel2.org как хост по умолчанию.
\item Маршруты хранятся в отдельных переменных, а не в одной куче как
    в примере \file{sample\_conf.py}.
\item Первый маршрут --- Dir --- обслуживает файлы из директории
    \file{tests} и использует \file{index.html} в качестве индексного файла.
\item Создаём прокси и направляем его на локальный веб-сервер; удобно
    для тестирования \ident{Proxy}-маршрута.
\item Далее идет \ident{Dir} для \url{http://mongrel2.org:6767/chatdemo/}. Посмотрим позже.
\item Также для демо-чата создаём \ident{Handler}, который и реализует всю логику.
\item После этого следует небольшой \ident{Handler} для тестирования
    обращений к обработчикам. Обратите внимание, несмотря на то, что они
    используют разные протоколы (демо-чат использует JSSockets) не нужно
    это явно указывать. Mongrel2 определяет это не из конфигурации, а по
    поведению.
\item Теперь, имея все переменные, мы можем создать хост
\ident{mongrel2}. Обратите внимание, я поленился выносить настройки
    для mp3stream-демо в отдельную переменную, и объявил их прямо в хэш-таблице. Это вполне
    допустимо. Запомните также, что можно использовать строки вида
    \verb|'blah'|, чтобы не экранировать обратные слэши.
\item Добавляем все хосты из переменной \ident{mongrel2} в \ident{main}-сервер.
\item Переменной \ident{settings} присваимваем массив глобальных
    настроек. В данном случае мы только увеличиваем количество рабочих
    потоков, которые 0MQ использует для своих нужд.
\item Наконец, заносим конфигурацию в базу.
\end{enumerate}

И это, пожалуй, самая сложная конфигурация на сегодняшний день.


\section{Маршрутизация и паттерны}

Код для работы с паттернами был взят из
\href{http://www.lua.org/}{Lua} и это, пожалуй, самый быстрый код
такого рода. Он очень похож на регулярные выражения, за исключением
того, что из него выкинуто всё лишнее и не нужное для маршрутизации
запросов. Также, в отличие от регулярных выражений, паттерны для
URLов всегда сравниваются с начала строки.

Вот как Mongrel2 ищет подходящий маршрут для заданного URLа. Он
разбивает маршруты на две части: префикс и паттерн. Затем ищет среди
маршрутов подходящий с самым длинным префиксом и анализирует паттерн.
Если паттерн подходит --- маршрут найден. Если паттерна нет вообще, то
маршрут снова подходит, поскольку фиксированная часть маршрута ---
префикс --- подошла.

Единственный нюанс, о котором надо помнить --- паттерны надо брать в
скобки. Собственно, здесь скобки используются исключительно для этих
целей --- отделять паттерн (в отличие от регулярных выражений, где они
обозначают группу). Таким образом, например, вместо
\shell{/images/.*.jpg} надо писать \shell{/images/(.*.jpg)}

Ниже приведён более детальный список символов, которые можно
использовать в паттернах:

\begin{itemize}
\item \verb|.| (точка) Все символы.
\item \verb|\a| Буквы.
\item \verb|\c| Управляющие символы.
\item \verb|\d| Цифры.
\item \verb|\l| Строчные буквы.
\item \verb|\p| Символы пунктуации.
\item \verb|\s| Пробелы.
\item \verb|\u| Заглавные буквы.
\item \verb|\w| Буквы и цифры.
\item \verb|\x| Шестнадцатиричные цифры.
\item \verb|\z| Символ окончания строки (null terminator).
\item \verb|[set]| Как и в регулярных выражениях --- [] обозначает
множество символом, например, [0-9] --- все цифры.
\item \verb|[^set]| Обратное множество, т.е. \verb|[^0-9]| --- всё что
угодно, кроме цифр.
\item \verb|*| Максимально возможное вхождение предыдущего символа; от
0 и больше.
\item \verb|+| Максимально возможное вхождение предыдущего символа; от
1 и больше.
\item \verb|-| Минимально возможное вхождение предыдущего символа; от
0 и больше.
\item \verb|?| 0 или 1 предыдущий символ.
\item \verb|\b|\emph{xy} Сбалансированное сопоставление подстроки,
начинающейся с \emph{x} и заканчивающейся на \emph{y}. Паттерн
\verb|\b()|, например, сопоставляет сбалансированные скобки.
\item \verb|$| Конец строки.
\end{itemize}

Экранированный символ в верхнем регистре работает с точностью до
наоборот, в отличие от своего младшего брата в нижнем. Т.е. \verb|\A| сопоставляет
любой символ, который \emph{не является} буквой. Обратный слэш
экранирует следующий за ним символ, отменяя, таким образом, его
специальные возможности (например, \verb|\\| сопоставляет бэкслэш).

Всё что здесь не упомянуто сопоставляется как есть.

\begin{aside}{Извините, юникодеры, но здесь всё в ASCII}
Да, я понимаю, всё должно быть в UTF-8 или какой другой разновидности
юникода. И вас, видимо, раздражает тот факт, что вы не можете
создавать URLы на том же языке, на котором говорите.

Что я могу сказать на это? Протоколы --- достаточно сложно
разрабатывать и если бы вам пришлось писать свой протокол, то,
вероятно, вы бы не хотели в конечном итоге иметь противоречивый
формат, который имеет несколько интерпретаций, который иногда сложно
переводить в верхний или нижний регистр, и который нужно транслировать
каждый раз, чтобы что-то с ним сделать. Поэтому юникоду здесь не
место. С ASCII жизнь сильно упрощается --- каждый компьютер просто
знает что это такое и с чем его едят.

Вот почему в интернете нередко приходится транслировать URLв в ASCII;
например, кодировать с помощью \%. Так говорит нам стандарт и это
весьма разумная вещь. Честно, я не хочу знать в чём разница между
крышечкой, двоеточием и прочими акцентами. Я хочу работать с
фиксированным набором символов. Не вините меня или Mongrel2 в этом;
это стандарт и, к тому же, хороший способ получить в итоге стабильный
сервер.

Протоколы работают тем лучше, чем меньше в них политики. Это значит,
что вы не можеет использовать юникод в паттернах. То есть вы можете,
конечно, но никто не гарантирует стабильный результат.

\end{aside}

Вот примеры маршрутов для изучения --- они помогут вам прочувствовать
систему немного лучше:

\begin{itemize}
\item \verb|"/images/"|  Сопоставляет любой путь, который содержит
/images/ без всяких паттернов.
\item \verb|"/"| Самый быстрый маршрут из всех возможных.
\item \verb|"/images/(.*.jpg)"| Сопоставляет картинки в формате jpg в
папке images. Обратите внимание на то, что паттерн не смотрит внутрь
директории, он только проверят паттерн \ident{(.*.jpg)}.
\item \verb|"/images/(\a-\-\d+\.jpg)"| Пример посложнее: он
сопоставляет короткую последовательность из 0 или больше букв (помните
-); затем тире (\verb|\-| экранирует -); далее длинная
последовательносить из 1 и больше цифр; и, наконец, .jpg, где
\verb|\|. экранирует точку.
\end{itemize}

Этого должно быть достаточно, чтобы понять как использовать паттерны.

\section{Логи}

m2sh ведёт логи всех операций, выполненных с помощью этой утилиты.
Кроме этого, с помощью команды \shell{m2sh commit} можно добавлять
свои комментарии. Все эти логи сохраняются, даже если вы выполняете
загрузку из скрипта (\shell{m2sh load}). В логах хранится подробная
информация: кто и что сделал с конфигурацией, когда это было сделано и
для какого сервера.

Чтобы посмотреть логи для нашего маленького примера, запустите
\shell{m2sh log -db simple.sqlite}.

Ниже --- пример для mongrel2.org:

\begin{code}{Пример логов}
\begin{lstlisting}
> m2sh log
[2010-07-18T04:14:53, mongrel2@zedshaw, init_command] /usr/bin/m2sh init
[2010-07-18T04:15:06, mongrel2@zedshaw, load_command] /usr/bin/m2sh load
[2010-07-18T04:22:06, mongrel2@zedshaw, load_command] /usr/bin/m2sh load
[2010-07-18T04:23:32, mongrel2@zedshaw, load_command] /usr/bin/m2sh load
[2010-07-18T04:26:16, mongrel2@zedshaw, upgrade] Latest code for Mongrel2.
[2010-07-18T18:05:59, mongrel2@zedshaw, load_command] /usr/bin/m2sh load
[2010-07-18T20:09:01, mongrel2@zedshaw, init_command] /usr/bin/m2sh config
[2010-07-18T20:09:02, mongrel2@zedshaw, load_command] /usr/bin/m2sh config
> m2sh commit -what mongrel2.org -why "Testing things out."
\end{lstlisting}
\end{code}

На сегодняшний день наблюдается тенденция хранить серверные
конфигурации в системах управления версиями, такими как git или
mercurial. Эти системы хранят историю изменений, но ничего не говорят
о причинах изменений. Логи восполняют этот пробел; собственно поэтому
эта фича и была добавлена в m2sh.

В будущем Mongrel2 будет вести статистику, которая привяжет изменения
в конфигурации с негативными изменениями в поведении сервера: частые
сбои, ошибки, медленная работа и прочие проблемы.

В общем, всё тайное становится явным. Mongrel2 позволит админам
вычислить, кто ``уронил'' Твиттер и кого надо за это уволить :-)

\section{Порт управления}

Перед самым релизом 1.0 мы добавили фичу под названием ``порт
управления'', которая позволяет соединиться с сервером через unix-сокет
и посылать команды. Вот что можно делать с помощью этих команд:
получать статус запущенных задач, получать список открытых соединений
и узнать как долго они открыты, принудительно закрыть соединение. С
помощью этого порта можно реализовать любой инструмент мониторинга.

По умолчанию этот порт находится в файле \file{run/control} в рабочей
директории (та, которую веб-сервер делает рутовой, помните?), но можно
изменить этот путь с помощью настройки \ident{control\_port}. Можно
использовать любую спецификацю ZeroMQ, но в целях безопасности я
рекомендую использовать IPC.

Как только Mongrel2 запущен, можно подключиться к нему с помощью
\file{m2sh} и управлять сервером с помощью простых команд. На данный
момент сервер возвращает не очень читабельный текст, но со временем мы
его улучшим.

Ниже приведён список команд:

\begin{description}
\item[status tasks] Возвращает список задач в JSON-формате. Что-то вроде
    внутренней команды \shell{ps}.
\item[status net] Возвращает список идентификаторов соединений и время
    (в секундах) с момента последней активности: если это HTTP-соединение,
    то время с момента подключения; если же сокет --- время с момента
    последнего пинга.
\item[time] Выводит время с сервера. Полезно для синхронизации.
\item[kill ID] Принудительно закрывает сокет с заданным
    идентификатором (его можно получить из \ident{status net}). Это
    достаточно жёсткий способ, поэтому используйте его как можно реже.
\item[control stop] Закрывает порт управления. Открыть сможет только
    перезагрузка сервера.
\end{description}

И пример того, как управлять с помощью \ident{m2sh}:

\begin{Verbatim}
m2sh control -db config.sqlite -name test
CONNECTING...
> status net
{"total": 0}
> time
{"time": 1282980306}
\end{Verbatim}

Эта фича ещё в разработке и со временем у неё будет более удобный
интерфейс.

Если вы хотите вручную соединиться и управлять сервером, то вот пример
на Питоне:


\begin{code}{Управление сервером: пример на Питоне}
\begin{lstlisting}
import zmq

CTX = zmq.Context()

addr = "ipc://run/control"

ctl = CTX.socket(zmq.REQ)

print "CONNECTING"
ctl.connect(addr)

while True:
    cmd = raw_input("> ")

    ctl.send(cmd)

    print ctl.recv()

ctl.close()
\end{lstlisting}
\end{code}

\section{Несколько серверов}

Процесс Mongrel2 не может хостить несколько серверов; при запуске вы
передаёте два параметра: базу настроек и UUID сервера, который надо
запустить. Такой подход упрощает код и позволяет поддерживать его в
рабочем состоянии.

Однако.

Утилита m2sh позволяет запускать несколько серверов из одной и той же
конфигурационной базы. Передавая параметр \ident{-every} вы сообщаете
о том, что надо выполнять действия над всеми серверами, определёнными
в настройках. Наряду с этим можно указать один сервер с помощью
идентификатора, хоста или названия. Если передаваемый параметр нельзя
трактовать однозначно (например, вы ищите с помощью \shell{-host
localhost} и вашей базе задано два сервера, которые обслуживают этот
домен), то m2sh выведет список соответствующих серверов и попросит
выбрать один.

Например:

\begin{Verbatim}
> m2sh start -db config.sqlite -every
Launching server localhost 9f0cbd7d-aeff-4195-921e-2ce1c25512d3 on port 6768
...
Launching server localhost 3d815ade-9081-4c36-94dc-77a9b060b021 on port 6767
...

> m2sh start -db config.sqlite -host localhost
Not sure which server to run, what I found:
NAME HOST UUID
--------------
localhost localhost 3d815ade-9081-4c36-94dc-77a9b060b021
localhost localhost 9f0cbd7d-aeff-4195-921e-2ce1c25512d3
* Use -every to run them all.

> m2sh start -db config.sqlite -uuid 3d815ade-9081-4c36-94dc-77a9b060b021
Launching server localhost 3d815ade-9081-4c36-94dc-77a9b060b021 on port 6767
...

> m2sh running -db config.sqlite -every
Found server localhost 3d815ade-9081-4c36-94dc-77a9b060b021 RUNNING at PID 28525
PID file run/mongrel2.pid not found for server localhost 9f0cbd7d-aeff-4195-921e-2ce1c25512d3

> m2sh stop -db config.sqlite -every
\end{Verbatim}

\section{Продвинутые настройки}

Mongrel2 можно конфигурировать с помощью настроек. Некоторые из них
нужно менять осторожно, поэтому тестируйте, прежде чем внести
изменения в рабочую систему. Все они имеют приемлемые значения по
умолчанию, так что, возможно, ничего и не надо менять, если нет острой
необходимости.

Чтобы настройки вступили в силу, их нужно передать в команду
\ident{commit}, при сохранени конфигурации, как в этом примере:

\begin{code}{Изменение настроек}
\begin{lstlisting}
settings = {"zeromq.threads": 1, "limits.url_path": 1024}
mimetypes = {".txt": "text/superawesome"}

servers = [main]
\end{lstlisting}
\end{code}

Mongrel2 считывает настройки на лету и для каждой из них создаёт
INFO-сообщение в логах со значением переменной --- это облегчает
отладку, в случае возникновения проблем.

Итак, полный список настроек:

\begin{description}
\item[control\_port=ipc://run/control] Mongrel2 слушает этот порт с
помощью 0MQ на предмет управляющих команд. Вам следует использовать
\verb|ipc://| для этих целей, чтобы только локальные пользователи
могли управлять сервером.
\item[limits.buffer\_size=2 * 1024] Внутренний буфер для ввода-вывода.
Используется для обработки запросов и проксирования. Это \emph{очень}
консервативное значение, так что если сервер получает HTTP-запросы
больше этого размера, то его надо увеличить. Возможно вы захотите
пристрелить того, кто посылает такие запросы: в среднем запрос обычно
не привышает 400-600 байт.
\item[limits.connection\_stack\_size=32 * 1024] Размер стэка для
корутин.
\item[limits.content\_length=20 * 1024] Максимальный размер контента,
который сервер принимает от клиентов. Сейчас это жёстко прописанный
лимит и все запросы, превышающие его, отвергаются. Будущие версии
позволят загружать файлы любого размера.
\item[limits.dir\_max\_path=256] Максимальная длина пути в
Dir-обработчиках.
\item[limits.dir\_send\_buffer=16 * 1024] Максимальный размер буфера
для пересылки файлов.
\item[limits.fdtask\_stack=100 * 1024] Размер стека подсистемы
управления соединениями и состояниями сервера. Это сердце веб-сервера
и, как в любом организме, оно одно, поэтому для него можно выделить
побольше ресурсов.
\item[limits.fdtask\_stack=100 * 1024] Размер стэка для одного
обработчика. Поскольку их обычно не очень много, то можно установить
достаточно большое значение. 
\item[limits.hanlder\_targets=128] Максимальное количество
обработчиков в бэкенде для получения сообщений. Не очень разумно
устанавливать большое значение.
\item[limits.header\_count=128 * 10] Максимальное количество
заголовков в клиентском соединении.
\item[limits.host\_name=256] Максимальная длина хоста.
\item[limits.mime\_ext\_len=128] Максимальная длина MIME-типа.
\item[limits.url\_path=256] Максимальная длина URLа. Длина параметров
не включена, только длина пути.
\item[superpoll.hot\_dividend=4] Доля ``горячих'' соединений (т.е.
1/4, 1/8). Чем больше простаивающих соединений, тем выше должно быть
это число. И наоборот, чем больше активных соединений, тем оно ниже.
\item[superpoll.max\_fd=10 * 1024] Максимально возможное количество
открытых файлов. Не устанавливайте больше чем 64 * 1024, и учтите, что
серверу понадобится некоторое время, чтобы создать необходимые
структуры данных.
\item[upload.temp\_store=None] Единственная переменная без значения по
умолчанию. Если вы планируете обрабатывать большие запросы, то задайте
временную папку, в которую сервер может писать. В секции
\ref{sec:async_file_upload_demo} эта настройка рассматривается более подробно.
\item[zeromq.threads=1] Количество потоков для 0MQ. Осторожно! У нас
возникали проблемы, если установить большое значение.
\end{description}
